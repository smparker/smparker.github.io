%!TEX program = latexmk -xelatex
\documentclass{nihbiosketch}
\usepackage[version=4]{mhchem}
\usepackage{cite}
\usepackage{bibunits}

%------------------------------------------------------------------------------
% biblatex stuff copied from Filipp
%------------------------------------------------------------------------------
\makeatletter
\newlength{\bibhang}
\setlength{\bibhang}{1em}
\newlength{\bibsep}
 {\@listi \global\bibsep\itemsep \global\advance\bibsep by\parsep}
\newlist{bibsection}{itemize}{3}
\setlist[bibsection]{label=,leftmargin={\bibhang+\widthof{[9]}},%
        itemindent=-\bibhang,
        itemsep=\bibsep,parsep=\z@,partopsep=0pt,
        topsep=0pt}
\newlist{bibenum}{enumerate}{3}
\setlist[bibenum]{label=\textbf{\arabic*.},leftmargin={\bibhang+\widthof{[999]}},%
        itemindent=-\bibhang,
        itemsep=\bibsep,parsep=\z@,partopsep=0pt,
        topsep=0pt}
\let\oldendbibenum\endbibenum
\def\endbibenum{\oldendbibenum\vspace{-.6\baselineskip}}
\let\oldendbibsection\endbibsection
\def\endbibsection{\oldendbibsection\vspace{-.6\baselineskip}}
\makeatother
%------------------------------------------------------------------------------

\name{Parker, Shane Matthew}
\eracommons{}
\position{Assistant Professor of Chemistry}

\begin{document}
%------------------------------------------------------------------------------

\begin{education}
University of Florida               & B.S           & 05/2008  & Chemistry, Mathematics \\
Northwestern University             & Ph.D.         & 05/2014  & Theoretical Chemistry \\
University of California, Irvine    & Postdoctoral  & 06/2019  & Quantum Chemistry \\
\end{education}


\section{Personal Statement}

\begin{statement}
I have been captivated by the interplay of light and chemistry for my entire
career: light can initiate chemical reactions (photochemistry),
control the reactivity (optical control), and probe
the results (spectroscopy); conversely, chemical reactions can produce light
(chemiluminescence) and nanostructures can concentrate and guide light.
My interest in light-matter interactions in chemistry has driven my
evolution as an electronic structure theorist, with the overall goal of
developing methods that can explain and predict complex light-matter
interactions from first-principles. I have designed and implemented high
performance methods in several
different software packages (including Bagel and Turbomole). Furthermore, I am
actively developing a program to enable rapid development of new computational
methods and immediate incorporation into large-scale simulations.

Nothing motivates me more than important problems that are out of reach of
state-of-the-art first principles theory. I find this unacceptable and take it
upon myself to build the methods---from first-principles theory to high
performance implementation---to tackle these important problems. For example, to
accurately describe multiexcitonic states in molecular aggregates, I designed
and implemented the active-space decomposition method that enables calculations
on workstations that previously would have been impossible on supercomputers.
Similarly, to uncover photochemical mechanisms for reactions lacking atomistic
information, I established methodology for exploratory simulations that can
discover reaction mechanisms with no prior knowledge, akin to numerical experimentation.

The quantum chemical methods I will build will launch a new paradigm
in methodologies to simulate nonlinear spectroscopies and allow for
predictions of multiphoton absorption in complex molecules with unprecedented realism.
This supports my long term goal of simulating ultrafast photochemistry
along with spectroscopic observables.

\begin{bibunit}[nih]
\nocite*{Parker2013JCP021108,Parker2018JCTC807,Parker2019}
\renewcommand{\refname}{\vspace{-2em}}
\putbib[../bib/pubs] %$
\end{bibunit}

\end{statement}

%------------------------------------------------------------------------------
\section{Positions and Honors}

\subsection*{Positions and Employment}
\begin{datetbl}
2016--2019  & Fellow, Arnold O. Beckman Postdoctoral Fellow, University of California, Irvine, Irvine, CA \\
2019--  & Assistant Professor, Dept. of Chemistry, Case Western Reserve University, Cleveland, OH \\
\end{datetbl}

\subsection*{Other Experience and Professional Memberships}
\begin{datetbl}
2012--           & Member, American Chemical Society Association \\
2012--           & Member, American Physical Society Association \\
\end{datetbl}

\subsection*{Honors}
\begin{datetbl}
2008--2009      & Fulbright Fellow, Technische Universit\"{a}t M\"{u}nchen, Munich, Germany \\
2010--2012      & Dept. of Energy Office of Science Graduate Fellow, Northwestern University, Evanston, IL \\
2016--2019      & Arnold O. Beckman Postdoctoral Fellow, University of California, Irvine, Irvine, CA \\
\end{datetbl}

%------------------------------------------------------------------------------

\section{Contribution to Science}

\begin{enumerate}
  \item \begin{bibunit}[nih]
  \textbf{Optical control of molecular torsion.}
  A central challenge in high-accuracy molecular beam experiments is deriving highly detailed atomistic
  reaction mechanisms, because cross sections must be averaged over all possible orientations and
  available vibrational states of
  the molecules in the beam. I proposed and simulated schemes to optically control electronic
  properties, molecular chirality, and molecular conductance of biaryl compounds through laser control over
  intramolecular vibrational modes.\cite{Parker2011JCP224301}
  Torsion in biaryl compounds---molecules with two aromatic moieties connected by a bond about which the barrier
  to rotation is small---mediates the electronic coupling between the two rings in the molecule, thus controlling
  the electron transfer and transport rates, the absorption and emission spectra, and the
  molecule's chirality.
  By tuning the pulse parameters and
  polarization, I showed that absorption spectra can be shifted, that free internal rotation can be started or
  that the molecular chirality
  can be inverted.\cite{Parker2011JCP224301} This control was extended when, with the aid of optimal control theory,
  I designed ``deracemizing'' control pulses, i.e., control
  pulses that convert a racemic mixture into an enantiopure mixture.\cite{Parker2012MolPhys1941}
  Furthermore, I showed that optical control could be combined with mechanical force to achieve
  enhanced control over the conductivity of biphenyl-dithiol in a molecular junction.\cite{Parker2014NanoLett4587}
  This type of fine control over molecular conductance is especially interesting in the case of molecular electronics,
  where external control of conductance is critical to constructing molecular circuits.

  \renewcommand{\refname}{\vspace{-2em}}
  \putbib[../bib/pubs] %$
  \end{bibunit}

  \item \begin{bibunit}[nih]
  \textbf{Excited-state electronic structure of molecular aggregates.}
  To shed light on the detailed mechanism of electronic processes like singlet fission,
  I pioneered the active space decomposition (ASD) method, together
  with Prof. Toru Shiozaki at NU, which was specifically designed to describe electronic processes
  in molecular aggregates.\cite{Parker2013JCP021108} Within ASD, the total wavefunction of an aggregate is
  constructed from fragment local wavefunctions.\cite{Parker2013JCP021108} ASD naturally leads to
  compact model Hamiltonians that bridge first-principles methods with chemical intuition and huge
  computational speed ups such that calculations can be performed on workstation computers
  that otherwise would be impossible even on today's most powerful supercomputers.
  I used ASD to compute model Hamiltonians from first principles for singlet fission in tetracene and pentacene
  with unprecedented accuracy that confirmed the dominance of the charge-transfer mediated mechanism
  in the acenes.\cite{Parker2014JPCC12700}
  ASD was also used to build model Hamiltonians
  for processes such as electron/hole transfer, excitation energy transfer, and triplet excitation transfer.\cite{Parker2014JCTC3738}
  I further extended ASD to treat multiple sites using the density matrix renormalization group method
  to avoid the exponential scaling with respect to number of monomer units.\cite{Parker2014JCP211102}
  Thus, ASD has been established as a computationally efficient, broadly applicable tool to understand
  electronic processes in (possibly covalently bound) molecular fragments, thereby offering a solution
  to a long-standing conundrum to combine the accuracy and rigor of \textit{ab initio} methods with
  the interpretability of model approaches.

  \renewcommand{\refname}{\vspace{-2em}}
  \putbib[../bib/pubs] %$
  \end{bibunit}

  \item \begin{bibunit}[nih]
  \textbf{Nonlinear electronic spectroscopy: successes and pitfalls.}
  Nonlinear spectroscopy is a suite of powerful tools to characterize complex systems, investigate fundamental chemical
  physics, and drive chemistry. Widespread use of simulated nonlinear spectroscopic
  properties has been stymied, however, by the high cost of quadratic response implementations using hybrid density
  functionals.

  I published a highly resource-efficient implementation of the quadratic response function---including excited-state
  absorption (ESA), two-photon absorption (TPA), and second harmonic generation (SHG) amplitudes---in Turbomole.\cite{Parker2018JCTC807}
  The implementation pursued the dual purpose of enabling calculations of nonlinear
  properties of molecules with 100s of atoms and 1000s of basis functions using hybrid density functionals and of
  clarifying the consequences of the quadratic response function's incorrect pole structure.\cite{Parker2016JCP134105}

  In addition, I showed that quadratic response functions from approximate theories diverge
  unphysically,\cite{Parker2016JCP134105} raising additional questions about the reliability of nonlinear response theory.
  Using the new implementation, I showed that the incorrect pole structure of the TDDFT quadratic
  response function within the adiabatic approximation leads to serious deficiencies in related properties, including
  unphysical divergences in excited-state absorption spectra, overly resonant two-photon absorption spectra, and spurious
  poles in the dynamic hyperpolarizability.\cite{Parker2016JCP134105,Parker2018JCTC807}
  However, in cases where the spurious resonances can be avoided, I showed that
  semiquantitative results suitable for molecule screening or analysis can be obtained, thus facilitating computational
  design of nonlinear optically active or two-photon active materials with greater realism and for larger systems than was
  previously possible.

  \renewcommand{\refname}{\vspace{-2em}}
  \putbib[../bib/pubs] %$
  \end{bibunit}

  \item \begin{bibunit}[nih]
  \textbf{Nonadiabatic molecular dynamics with TDDFT.}
  Nonadiabatic molecular dynamics (NAMD) with time-dependent density functional theory (TDDFT) is
  emerging as a powerful technique to unravel the precise details of complex excited-state reactions.
  I contributed to the growth of the field by demonstrating its capacity for photoreactivity mechanism discovery
  and extending this paradigm to reactions involving more than one excited state.

  Together with colleague's during my postdoc, I performed the first fully unconstrained on-the-fly NAMD
  simulations of the first step of the oxygen evolution reaction on small hydrated \ce{TiO2} nanoclusters
  using surface hopping (SH) coupled with TDDFT.\cite{Muuronen2017ChemSci2179}
  Such on-the-fly simulations allow for truly exploratory mechanism \emph{discovery}, i.e., no
  prior knowledge of the mechanism is required. On the basis of our simulations, we proposed that the reaction is driven by
  electron-proton transfer (EPT) from physisorbed water to a photohole strongly localized on a bridging (\ce{Ti-O-Ti})
  oxygen atom. This proposed mechanism is consistent with the observed pH dependence as well as the
  observed production of mobile hydroxyl radicals and could not be observed computationally without the fully consistent
  model including exciton-binding. The identification of a detailed mechanism opens the door to designing photocatalysts
  with greatly improved efficiency.

  I also implemented nonadiabatic couplings between excited states, paving the way for photodynamics
  involving multiple excited states, such as the photodeactivation of thymine.\cite{Parker2019}
  In particular, I showed that excited-state lifetimes (153 fs for \ce{S2} and 14.1 ps for \ce{S1}) obtained
  from SH-TDDFT simulations are in excellent agreement with experimentally observed lifetimes (100--200 fs and
  5--7 ps); indeed, among all investigated electronic structure methods---including ADC(2) and CASSCF---only TDDFT produces
  qualitative agreement with experimental results, despite its low computational cost.
  This work thus adds to the mounting evidence that SH-TDDFT strikes an ideal cost-to-performance balance.

  \renewcommand{\refname}{\vspace{-2em}}
  \putbib[../bib/pubs] %$
  \end{bibunit}
\end{enumerate}

\subsection*{Complete List of Published Work:}
\url{https://quantumparker.com}

%------------------------------------------------------------------------------

\section{Research Support}

\subsection*{Ongoing Research Support}

\grantinfo{}{Parker (PI)}{07/01/19--06/31/2022}
{Startup}
{Provided by Case Western Reserve University}
{Role: PI}



%------------------------------------------------------------------------------

\subsection*{Completed Research Support}
%
% \grantinfo{R21 AA998075}{Hunt (PI)}{01/01/11--12/31/13}
% {Community-based intervention for alcohol abuse}
% {The goal of this project was to assess a community-based strategy for reducing alcohol abuse among
% older individuals.}
% {Role: PI}



\end{document}
