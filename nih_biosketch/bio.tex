%!TEX program = latexmk -xelatex
\documentclass{nihbiosketch}
\usepackage[version=4]{mhchem}
\usepackage{cite}
\usepackage{bibunits}

%------------------------------------------------------------------------------
% biblatex stuff copied from Filipp
%------------------------------------------------------------------------------
\makeatletter
\newlength{\bibhang}
\setlength{\bibhang}{1em}
\newlength{\bibsep}
 {\@listi \global\bibsep\itemsep \global\advance\bibsep by\parsep}
\newlist{bibsection}{itemize}{3}
\setlist[bibsection]{label=,leftmargin={\bibhang+\widthof{[9]}},%
        itemindent=-\bibhang,
        itemsep=\bibsep,parsep=\z@,partopsep=0pt,
        topsep=0pt}
\newlist{bibenum}{enumerate}{3}
\setlist[bibenum]{label=\textbf{\arabic*.},leftmargin={\bibhang+\widthof{[999]}},%
        itemindent=-\bibhang,
        itemsep=\bibsep,parsep=\z@,partopsep=0pt,
        topsep=0pt}
\let\oldendbibenum\endbibenum
\def\endbibenum{\oldendbibenum\vspace{-.6\baselineskip}}
\let\oldendbibsection\endbibsection
\def\endbibsection{\oldendbibsection\vspace{-.6\baselineskip}}
\makeatother
%------------------------------------------------------------------------------

\name{Parker, Shane Matthew}
\eracommons{}
\position{Assistant Professor of Chemistry}

\begin{document}
%------------------------------------------------------------------------------

\begin{education}
University of Florida               & B.S           & 05/2008  & Chemistry, Mathematics \\
Northwestern University             & Ph.D.         & 05/2014  & Theoretical Chemistry \\
University of California, Irvine    & Postdoctoral  & 06/2019  & Computational Chemistry \\
\end{education}


\section{Personal Statement}

\begin{statement}
% I have the expertise, leadership, training, expertise and motivation necessary
% to successfully carry out the proposed research project.  I have a broad
% background in psychology, with specific training and expertise in ethnographic
% and survey research and secondary data analysis on psychological aspects of
% drug addiction.  My research includes neuropsychological changes associated
% with addiction.  As PI or co-Investigator on several university- and NIH-
% funded grants, I laid the groundwork for the proposed research by developing
% effective measures of disability, depression, and other psychosocial factors
% relevant to the aging substance abuser, and by establishing strong ties with
% community providers that will make it possible to recruit and track
% participants over time as documented in the following publications.  In
% addition, I successfully administered the projects (e.g.\ staffing, research
% protections, budget), collaborated with other researchers, and produced several
% peer-reviewed publications from each project.  As a result of these previous
% experiences, I am aware of the importance of frequent communication among
% project members and of constructing a realistic research plan, timeline, and
% budget.  The current application builds logically on my prior work. During
% 2005--2006 my career was disrupted due to family obligations. However, upon
% returning to the field I immediately resumed my research projects and
% collaborations and successfully competed for NIH support.
% 
% \begin{enumerate}
% 
% \item Merryle, R.J. \& Hunt, M.C. (2004). Independent living, physical
%         disability and substance abuse among the elderly. Psychology and Aging,
%         23(4), 10--22.
% 
% \item Hunt, M.C., Jensen, J.L. \& Crenshaw, W. (2007). Substance abuse and
%         mental health among community-dwelling elderly. International Journal
%         of Geriatric Psychiatry, 24(9), 1124--1135.
% 
% \item Hunt, M.C., Wiechelt, S.A. \& Merryle, R. (2008). Predicting the
%         substance-abuse treatment needs of an aging population.  American
%         Journal of Public Health, 45(2), 236--245. PMCID: PMC9162292 
% 
% \item Hunt, M.C., Newlin, D.B. \& Fishbein, D. (2009). Brain imaging in
%         methamphetamine abusers across the life-span. Gerontology, 46(3),
%         122--145.
% 
% \end{enumerate}

\end{statement}

%------------------------------------------------------------------------------
\section{Positions and Honors}

\subsection*{Positions and Employment}
\begin{datetbl}
2016--2019  & Fellow, Arnold O. Beckman Postdoctoral Fellow, University of California, Irvine, Irvine, CA \\
2019--  & Assistant Professor, Dept. of Chemistry, Case Western Reserve University, Cleveland, OH \\
\end{datetbl}

\subsection*{Other Experience and Professional Memberships}
\begin{datetbl}
2012--           & Member, American Chemical Society Association \\
2012--           & Member, American Physical Society Association \\
\end{datetbl}

\subsection*{Honors}
\begin{datetbl}
2008--2009      & Fulbright Fellow, Technische Universit\"{a}t M\"{u}nchen, Munich, Germany \\
2010--2012      & Dept. of Energy Office of Science Graduate Fellow, Northwestern University, Evanston, IL \\
2016--2019      & Arnold O. Beckman Postdoctoral Fellow, University of California, Irvine, Irvine, CA \\
\end{datetbl}

%------------------------------------------------------------------------------

\section{Contribution to Science}

\begin{enumerate}

% \item My early publications directly addressed the fact that substance abuse is
%     often overlooked in older adults. However, because many older adults were
%     raised during an era of increased drug and alcohol use, there are reasons
%     to believe that this will become an increasing issue as the population
%     ages.   These publications found that older adults appear in a variety of
%     primary care settings or seek mental health providers to deal with emerging
%     addiction problems.  These publications document this emerging problem but
%     guide primary care providers and geriatric mental health providers to
%     recognize symptoms, assess the nature of the problem and apply the
%     necessary interventions.   By providing evidence and simple clinical
%     approaches, this body of work has changed the standards of care for
%     addicted older adults and will continue to provide assistance in relevant
%     medical settings well into the future.  I served as the primary
%     investigator or co-investigator in all of these studies. 
% 
  \item
  As a graduate student at NU, I proposed and simulated schemes to optically control electronic
  properties, molecular chirality, and molecular conductance of biaryl compounds through laser control over
  intramolecular vibrational modes.\cite{Parker2011JCP}
  Torsion in biaryl compounds---molecules with two aromatic moieties connected by a bond about which the barrier
  to rotation is small---mediates the electronic coupling between the two rings in the molecule, thus controlling
  the electron transfer and transport rates, the absorption and emission spectra, and the
  molecule's chirality.
  By tuning the pulse parameters and
  polarization, I showed that absorption spectra can be shifted, that free internal rotation can be started or
  that the molecular chirality
  can be inverted. This control was extended when, with the aid of optimal control theory,
  I designed ``deracemizing'' control pulses, i.e., control
  pulses that convert a racemic mixture into an enantiopure mixture.\cite{Parker2012MolPhys}

  In a further demonstration, motivated by the fascinating fundamental chemical physics of molecular junctions,
  I proposed the combination of single molecule pulling and optical control as a way to enhance control over the
  electron transport characteristics of a molecular junction.\cite{Parker:2014kt} I demonstrated this possibility using a model junction
  consisting of biphenyl-dithiol coupled to gold contacts (Fig. \ref{fig:pulling}). The junction is pulled while optically
  manipulating the dihedral angle between the two rings. Using quantum dynamics simulations and nonequilibrium Green's function
  density functional theory based transport calculations, I showed
  that the greatest degree of control was attained when optical control was combined with
  mechanical force.
  This type of fine control over molecular conductance is especially interesting in the case of molecular electronics,
  where external control of conductance is critical to constructing molecular circuits.
  
  \begin{bibunit}[nih]
    \renewcommand{\refname}{\vspace{-2em}}
    \nocite{Parker2011JCP224301,Parker2012MolPhys1941,Parker2014NanoLett4587}
    \putbib[../bib/pubs] %$
  \end{bibunit}

  \item
  Singlet fission\cite{Smith2010ChemRev} is a molecular process by which a photogenerated singlet exciton
  splits into two entangled triplet excitons. Through this carrier multiplication, singlet fission has
  the potential to significantly boost theoretical solar cell efficiencies up to 44\%, well beyond the
  single bandgap Shockley-Queisser limit
  of 34\%.\cite{Hanna2006JAP,Smith2010ChemRev} However, fundamental questions about the microscopic
  mechanism remain unanswered, impeding the development of stable and efficient singlet fission chromophores.

  To shed light on the detailed mechanism, I pioneered the active space decomposition (ASD) method, together
  with Prof. Toru Shiozaki at NU, which was specifically designed to describe electronic processes
  in molecular aggregates.\cite{Parker2013JCP} Within ASD, the total wavefunction of an aggregate is
  constructed from fragment local wavefunctions.\cite{Parker2013JCP} ASD naturally leads to
  compact model Hamiltonians that bridge first-principles methods with chemical intuition and huge
  computational speed ups such that calculations can be performed on workstation computers
  that otherwise would be impossible even on today's most powerful supercomputers.

  I used ASD to compute model Hamiltonians from first principles for singlet fission in tetracene and pentacene
  with unprecedented accuracy that confirmed the dominance of the charge-transfer mediated mechanism
  in the acenes.\cite{Parker2014JPCC}
  ASD was also used to build model Hamiltonians
  for processes such as electron/hole transfer, excitation energy transfer, and triplet excitation transfer.\cite{Parker2014JCTC}
  I further extended ASD to treat multiple sites using the density matrix renormalization group method
  to avoid the exponential scaling with respect to number of monomer units.\cite{Parker2014JCP}
  Recent developments have enabled the computation of ASD model Hamiltonians of the same processes
  for covalently bound chromophores.\cite{Kim2015JCTC}
  Thus, ASD has been established as a computationally efficient, broadly applicable tool to understand
  electronic processes in (possibly covalently bound) molecular fragments, thereby offering a solution
  to a long-standing conundrum to combine the accuracy and rigor of \textit{ab initio} methods with
  the interpretability of model approaches.

  \item
  Nonlinear spectroscopy is a suite of powerful tools to characterize complex systems, investigate fundamental chemical
  physics, and drive chemistry. As such, nonlinear response theory plays a central role in theoretical
  materials chemistry and applications of nonlinear response theory within time-dependent density functional theory
  (TDDFT) are increasingly important. Widespread use of simulated nonlinear spectroscopic
  properties has been stymied, however, by the high cost of quadratic response implementations using hybrid density
  functionals.
  In addition, as a postdoc at UCI, I showed that quadratic response functions from approximate theories diverge
  unphysically,\cite{Parker:2016ke} raising additional questions about the reliability of nonlinear response theory.

  I recently completed a highly resource-efficient implementation of the quadratic response function---including excited-state
  absorption (ESA), two-photon absorption (TPA), and second harmonic generation (SHG) amplitudes---in Turbomole.\cite{Parker2017quad}
  The implementation pursued the dual purpose of enabling calculations of nonlinear
  properties of molecules with 100s of atoms and 1000s of basis functions using hybrid density functionals and of
  clarifying the consequences of the quadratic response function's incorrect pole structure.\cite{Parker:2016ke} As
  an example, to demonstrate the efficiency of the implementation, I simulated two-photon absorption spectrum of a \ce{Ni}
  porphyrin tetramer with 446 atoms and 4548 basis functions.

  Using the new implementation, I showed that the incorrect pole structure of the TDDFT quadratic
  response function within the adiabatic approximation leads to serious deficiencies in related properties, including
  unphysical divergences in excited-state absorption spectra, overly resonant two-photon absorption spectra, and spurious
  poles in the dynamic hyperpolarizability. However, in cases where the spurious resonances can be avoided, I showed that
  semiquantitative results suitable for molecule screening or analysis can be obtained, thus facilitating computational
  design of nonlinear optically active or two-photon active materials with greater realism and for larger systems than was
  previously possible.

  \item
  Titania (\ce{TiO2}) nanoparticles are interesting redox photocatalytic materials due to their abundance,
  inexpensiveness, and non-toxicity. In the decades since titania was shown to act as a photocatalyst for water-splitting,
  titania has been the subject of intense research.\cite{Fujishima1972Nature} However, the efficiency of photocatalytic
  water-splitting remains limited, and the path to improvement is clouded by the lack of specific mechanistic details into
  the ultrafast electronically nonadiabatic processes.

  Together with colleague's during my postdoc, I performed the first fully unconstrained on-the-fly nonadiabatic molecular
  dynamics (NAMD) simulations of the first step of the oxygen evolution reaction on small hydrated \ce{TiO2} nanoclusters
  using Fewest Switches Surface Hopping (FSSH) coupled with time-dependent density functional theory
  (TDDFT).\cite{C6SC04378J} Such on-the-fly simulations allow for truly exploratory mechanism \emph{discovery}, i.e., no
  prior knowledge of the mechanism is required. On the basis of our simulations, we proposed that the reaction is driven by
  electron-proton transfer (EPT) from physisorbed water to a photohole strongly localized on a bridging (\ce{Ti-O-Ti})
  oxygen atom (Fig. \ref{fig:tio2}). This proposed mechanism is consistent with the observed pH dependence as well as the
  observed production of mobile hydroxyl radicals and could not be observed computationally without the fully consistent
  model including exciton-binding. The identification of a detailed mechanism opens the door to designing photocatalysts
  with greatly improved efficiency.
\end{enumerate}

\subsection*{Complete List of Published Work:} 
\url{https://quantumparker.com}


%------------------------------------------------------------------------------

\section{Research Support}

\subsection*{Ongoing Research Support}

\grantinfo{}{Parker (PI)}{07/01/19--06/31/2022}
{Startup}
{Provided by Case Western Reserve University}
{Role: PI}



%------------------------------------------------------------------------------

\subsection*{Completed Research Support}
% 
% \grantinfo{R21 AA998075}{Hunt (PI)}{01/01/11--12/31/13}
% {Community-based intervention for alcohol abuse}
% {The goal of this project was to assess a community-based strategy for reducing alcohol abuse among 
% older individuals.}
% {Role: PI}



\end{document}
